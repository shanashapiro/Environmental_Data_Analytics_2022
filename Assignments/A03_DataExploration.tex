% Options for packages loaded elsewhere
\PassOptionsToPackage{unicode}{hyperref}
\PassOptionsToPackage{hyphens}{url}
%
\documentclass[
]{article}
\title{Assignment 3: Data Exploration}
\author{Student Name, Section \#0}
\date{}

\usepackage{amsmath,amssymb}
\usepackage{lmodern}
\usepackage{iftex}
\ifPDFTeX
  \usepackage[T1]{fontenc}
  \usepackage[utf8]{inputenc}
  \usepackage{textcomp} % provide euro and other symbols
\else % if luatex or xetex
  \usepackage{unicode-math}
  \defaultfontfeatures{Scale=MatchLowercase}
  \defaultfontfeatures[\rmfamily]{Ligatures=TeX,Scale=1}
\fi
% Use upquote if available, for straight quotes in verbatim environments
\IfFileExists{upquote.sty}{\usepackage{upquote}}{}
\IfFileExists{microtype.sty}{% use microtype if available
  \usepackage[]{microtype}
  \UseMicrotypeSet[protrusion]{basicmath} % disable protrusion for tt fonts
}{}
\makeatletter
\@ifundefined{KOMAClassName}{% if non-KOMA class
  \IfFileExists{parskip.sty}{%
    \usepackage{parskip}
  }{% else
    \setlength{\parindent}{0pt}
    \setlength{\parskip}{6pt plus 2pt minus 1pt}}
}{% if KOMA class
  \KOMAoptions{parskip=half}}
\makeatother
\usepackage{xcolor}
\IfFileExists{xurl.sty}{\usepackage{xurl}}{} % add URL line breaks if available
\IfFileExists{bookmark.sty}{\usepackage{bookmark}}{\usepackage{hyperref}}
\hypersetup{
  pdftitle={Assignment 3: Data Exploration},
  pdfauthor={Student Name, Section \#0},
  hidelinks,
  pdfcreator={LaTeX via pandoc}}
\urlstyle{same} % disable monospaced font for URLs
\usepackage[margin=2.54cm]{geometry}
\usepackage{graphicx}
\makeatletter
\def\maxwidth{\ifdim\Gin@nat@width>\linewidth\linewidth\else\Gin@nat@width\fi}
\def\maxheight{\ifdim\Gin@nat@height>\textheight\textheight\else\Gin@nat@height\fi}
\makeatother
% Scale images if necessary, so that they will not overflow the page
% margins by default, and it is still possible to overwrite the defaults
% using explicit options in \includegraphics[width, height, ...]{}
\setkeys{Gin}{width=\maxwidth,height=\maxheight,keepaspectratio}
% Set default figure placement to htbp
\makeatletter
\def\fps@figure{htbp}
\makeatother
\setlength{\emergencystretch}{3em} % prevent overfull lines
\providecommand{\tightlist}{%
  \setlength{\itemsep}{0pt}\setlength{\parskip}{0pt}}
\setcounter{secnumdepth}{-\maxdimen} % remove section numbering
\ifLuaTeX
  \usepackage{selnolig}  % disable illegal ligatures
\fi

\begin{document}
\maketitle

\hypertarget{overview}{%
\subsection{OVERVIEW}\label{overview}}

This exercise accompanies the lessons in Environmental Data Analytics on
Data Exploration.

\hypertarget{directions}{%
\subsection{Directions}\label{directions}}

\begin{enumerate}
\def\labelenumi{\arabic{enumi}.}
\tightlist
\item
  Change ``Student Name, Section \#'' on line 3 (above) with your name
  and section number.
\item
  Work through the steps, \textbf{creating code and output} that fulfill
  each instruction.
\item
  Be sure to \textbf{answer the questions} in this assignment document.
\item
  When you have completed the assignment, \textbf{Knit} the text and
  code into a single PDF file.
\item
  After Knitting, submit the completed exercise (PDF file) to the
  dropbox in Sakai. Add your last name into the file name (e.g.,
  ``FirstLast\_A03\_DataExploration.Rmd'') prior to submission.
\end{enumerate}

The completed exercise is due on \textless\textgreater.

\hypertarget{set-up-your-r-session}{%
\subsection{Set up your R session}\label{set-up-your-r-session}}

\begin{enumerate}
\def\labelenumi{\arabic{enumi}.}
\tightlist
\item
  Check your working directory, load necessary packages (tidyverse), and
  upload two datasets: the ECOTOX neonicotinoid dataset
  (ECOTOX\_Neonicotinoids\_Insects\_raw.csv) and the Niwot Ridge NEON
  dataset for litter and woody debris
  (NEON\_NIWO\_Litter\_massdata\_2018-08\_raw.csv). Name these datasets
  ``Neonics'' and ``Litter'', respectively. \textbf{Be sure to add the
  \texttt{stringsAsFactors\ =\ TRUE} parameter to the function when
  reading in the CSV files.}
\end{enumerate}

\hypertarget{learn-about-your-system}{%
\subsection{Learn about your system}\label{learn-about-your-system}}

\begin{enumerate}
\def\labelenumi{\arabic{enumi}.}
\setcounter{enumi}{1}
\tightlist
\item
  The neonicotinoid dataset was collected from the Environmental
  Protection Agency's ECOTOX Knowledgebase, a database for ecotoxicology
  research. Neonicotinoids are a class of insecticides used widely in
  agriculture. The dataset that has been pulled includes all studies
  published on insects. Why might we be interested in the ecotoxicologoy
  of neonicotinoids on insects? Feel free to do a brief internet search
  if you feel you need more background information.
\end{enumerate}

\begin{quote}
Answer:
\end{quote}

\begin{enumerate}
\def\labelenumi{\arabic{enumi}.}
\setcounter{enumi}{2}
\tightlist
\item
  The Niwot Ridge litter and woody debris dataset was collected from the
  National Ecological Observatory Network, which collectively includes
  81 aquatic and terrestrial sites across 20 ecoclimatic domains. 32 of
  these sites sample forest litter and woody debris, and we will focus
  on the Niwot Ridge long-term ecological research (LTER) station in
  Colorado. Why might we be interested in studying litter and woody
  debris that falls to the ground in forests? Feel free to do a brief
  internet search if you feel you need more background information.
\end{enumerate}

\begin{quote}
Answer:
\end{quote}

\begin{enumerate}
\def\labelenumi{\arabic{enumi}.}
\setcounter{enumi}{3}
\tightlist
\item
  How is litter and woody debris sampled as part of the NEON network?
  Read the NEON\_Litterfall\_UserGuide.pdf document to learn more. List
  three pieces of salient information about the sampling methods here:
\end{enumerate}

\begin{quote}
Answer: \emph{ } *
\end{quote}

\hypertarget{obtain-basic-summaries-of-your-data-neonics}{%
\subsection{Obtain basic summaries of your data
(Neonics)}\label{obtain-basic-summaries-of-your-data-neonics}}

\begin{enumerate}
\def\labelenumi{\arabic{enumi}.}
\setcounter{enumi}{4}
\item
  What are the dimensions of the dataset?
\item
  Using the \texttt{summary} function on the ``Effect'' column,
  determine the most common effects that are studied. Why might these
  effects specifically be of interest?
\end{enumerate}

\begin{quote}
Answer:
\end{quote}

\begin{enumerate}
\def\labelenumi{\arabic{enumi}.}
\setcounter{enumi}{6}
\tightlist
\item
  Using the \texttt{summary} function, determine the six most commonly
  studied species in the dataset (common name). What do these species
  have in common, and why might they be of interest over other insects?
  Feel free to do a brief internet search for more information if
  needed.
\end{enumerate}

\begin{quote}
Answer:
\end{quote}

\begin{enumerate}
\def\labelenumi{\arabic{enumi}.}
\setcounter{enumi}{7}
\tightlist
\item
  Concentrations are always a numeric value. What is the class of
  Conc.1..Author. in the dataset, and why is it not numeric?
\end{enumerate}

\begin{quote}
Answer:
\end{quote}

\hypertarget{explore-your-data-graphically-neonics}{%
\subsection{Explore your data graphically
(Neonics)}\label{explore-your-data-graphically-neonics}}

\begin{enumerate}
\def\labelenumi{\arabic{enumi}.}
\setcounter{enumi}{8}
\item
  Using \texttt{geom\_freqpoly}, generate a plot of the number of
  studies conducted by publication year.
\item
  Reproduce the same graph but now add a color aesthetic so that
  different Test.Location are displayed as different colors.
\end{enumerate}

Interpret this graph. What are the most common test locations, and do
they differ over time?

\begin{quote}
Answer:
\end{quote}

\begin{enumerate}
\def\labelenumi{\arabic{enumi}.}
\setcounter{enumi}{10}
\tightlist
\item
  Create a bar graph of Endpoint counts. What are the two most common
  end points, and how are they defined? Consult the ECOTOX\_CodeAppendix
  for more information.
\end{enumerate}

\begin{quote}
Answer:
\end{quote}

\hypertarget{explore-your-data-litter}{%
\subsection{Explore your data (Litter)}\label{explore-your-data-litter}}

\begin{enumerate}
\def\labelenumi{\arabic{enumi}.}
\setcounter{enumi}{11}
\item
  Determine the class of collectDate. Is it a date? If not, change to a
  date and confirm the new class of the variable. Using the
  \texttt{unique} function, determine which dates litter was sampled in
  August 2018.
\item
  Using the \texttt{unique} function, determine how many plots were
  sampled at Niwot Ridge. How is the information obtained from
  \texttt{unique} different from that obtained from \texttt{summary}?
\end{enumerate}

\begin{quote}
Answer:
\end{quote}

\begin{enumerate}
\def\labelenumi{\arabic{enumi}.}
\setcounter{enumi}{13}
\item
  Create a bar graph of functionalGroup counts. This shows you what type
  of litter is collected at the Niwot Ridge sites. Notice that litter
  types are fairly equally distributed across the Niwot Ridge sites.
\item
  Using \texttt{geom\_boxplot} and \texttt{geom\_violin}, create a
  boxplot and a violin plot of dryMass by functionalGroup.
\end{enumerate}

Why is the boxplot a more effective visualization option than the violin
plot in this case?

\begin{quote}
Answer:
\end{quote}

What type(s) of litter tend to have the highest biomass at these sites?

\begin{quote}
Answer:
\end{quote}

\end{document}
